\subsection{Упражнение 1}

Реализовать алгоритм БПФ.

Возьмём простой массив для примера:

\begin{lstlisting}[language=Python]
ys = [0.9,0.7,-0.6,-0.6]
hs = np.fft.fft(ys)
\end{lstlisting}
\begin{lstlisting}
array([0.4+0.j , 1.5-1.3j, 0.2+0.j , 1.5+1.3j])
\end{lstlisting}

\begin{lstlisting}[language=Python]
def dft(ys):
    N = len(ys)
    ts = np.arange(N) / N
    freqs = np.arange(N)
    args = np.outer(ts, freqs)
    M = np.exp(1j * PI2 * args)
    amps = M.conj().transpose().dot(ys)
    return amps
\end{lstlisting}

Далее разделим массив на элементы:

\begin{lstlisting}[language=Python]
def my_fft(ys):
  He = dft(ys[::2])
  Ho = dft(ys[1::2])
  ns = np.arange(len(ys))
  W = np.exp(-1j * PI2 * ns / len(ys))
  return np.tile(He, 2) + W * np.tile(Ho, 2)
\end{lstlisting}

И добавим рекурсивный вызов:

\begin{lstlisting}[language=Python]
def my_fft(ys):
  if len(ys) == 1:
    return ys
  He = my_fft(ys[::2])
  Ho = my_fft(ys[1::2])
  ns = np.arange(len(ys))
  W = np.exp(-1j * PI2 * ns / len(ys))
  return np.tile(He, 2) + W * np.tile(Ho, 2)
\end{lstlisting}

Таким образом, мы написали собственную функцию БПФ. Попробуем её на нашем массиве:

\begin{lstlisting}[language=Python]
my_fft(ys)
\end{lstlisting}
\begin{lstlisting}
array([0.4+0.0000000e+00j, 1.5-1.3000000e+00j, 0.2-1.2246468e-17j,
       1.5+1.3000000e+00j])
\end{lstlisting}

Результат идентичен с библиотечной функцией.

\subsection{Вывод}

Дискретное преобразование Фурье  — это одно из преобразований Фурье, широко применяемых в алгоритмах цифровой обработки сигналов , а также в других областях, связанных с анализом частот в дискретномсигнале. Дискретное преобразование Фурье требует в качестве входа дискретную функцию. Такие функции часто создаются путём дискретизации. В качестве упражнения была написана одна из реализаций БПФ.